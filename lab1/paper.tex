\documentclass{article}

\usepackage{listings}


\lstset{
  extendedchars=\true,
  inputencoding=utf8,
  language=Matlab,
  %showstringspaces=false,
  formfeed=\newpage,
  tabsize=4,
  %commentstyle=\itshape,
  basicstyle=\ttfamily\scriptsize,
  %basicstyle={\small\fontfamily{fvm}\fontseries{m}\selectfont},
  commentstyle=\color{Apricot}\bfseries,
  %commentstyle=\color{red}\itshape,
  stringstyle=\color{red},
  identifierstyle=\color{PineGreen},
  showstringspaces=false,
  keywordstyle=\color{blue}\bfseries,
  moredelim=[il][\large\textbf]{\#\# },
  morekeywords={models,range},
  numbers=left,
  numbersep=2pt,
  numberstyle=\tiny,%\color{blue}\bfseries,
  literate=%
  {ã}{{\~a}}1
  {â}{{\^a}}1
  {õ}{{\~o}}1
  {á}{{\'a}}1
  {ú}{{\'u}}1
  {í}{{\'i}}1
  {é}{{\'e}}1
  {Ç}{{\c{C}}}1
  {Õ}{{\~O}}1
  {Ê}{{\^E}}1
  {ó}{{\'o}}1
  {à}{{\`a}}1
  {Â}{{\^A}}1
  {ô}{{\^o}}1
  {ê}{{\^e}}1
  {ç}{{\c{c}}}1
}



\begin{document}

\title{Computer Vision Lab \#1: Pointwise Image Processing} 

\author{Ricardo Fabbri and Others \and Based on EN161 Image
Understanding 2011 course material from Brown University}


Your goal of this first lab is to 

\begin{itemize}
\item Familiarize with the Scilab image
processing toolbox (SIP).
\item Implement pointwise image processing operations such as thresholding,
sampling, quantization, and enhancements
\end{itemize}

The images you will need for the lab can be downloaded from the course
website
http://wiki.nosdigitais.teia.org.br/CV

\section{Introduction to the Scilab Image Processing toolbox (SIP)}

\paragraph{Initial tasks:}
\begin{enumerate}
\item Install Scilab and SIP. You will need Linux or Mac OSX.
\item Go through the SIP demos and make sure you understand the \texttt{imshow, imread,
imwrite}, and the other commands showcased in the demo.
\begin{lstlisting}
exec(SIPDEMO);
\end{lstlisting}
\item Read our paper about the toolbox and type in the commands~\cite{SIP}.
Make things work in the desired way even if they don't, since your version of
SIP and Scilab may be different.
\end{enumerate}

\section{Thresholding}

\subsection{Basics} Using the white blood cell image1 as seen below, threshold it to make any
non-cell area black. To do this, you will need to check each pixel in the image,
and if it is over $\tau_0$ (where $\tau_0$ is the threshold you have chosen) set the pixel’s
intensity to 0, otherwise, set to 1. You will need to play with $\tau_0$ to find its
optimal value. You will probably want to pass it into the function as a
parameter.
\code{Threshold Code}{threshold-code.sci}
\subsection{Bilevel Thresholding}
Using the snowman image 2, perform bilevel thresholding. i.e. set all
pixels with intensity between $\tau_0$ and $\tau_1$ to 255 and set all pixels
less than $\tau_0$ 
to 0 and all pixels greater than $\tau_1$ to 0. The snowman is the region of interest,
you are thresholding for (where bob is, should be all white and the rest should
be black).


\end{document}
